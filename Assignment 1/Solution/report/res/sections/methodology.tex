\section{Methodology}

\subsection{Model Preparation} \label{subsec:modprep}

To work with any SMV Models, the loaded .smv file needs to be converted into type \texttt{BddFsm}, which is a Python class for finite state machine-like (FSM-like) structures, encoded into BDDs. As the SMV model is loaded into the global environment, this can be done by calling $\verb|pynusmv.glob.prop_database().master.bddFsm|$. The $\verb|prop|$ library provides a method $\verb|prop.expr|$ to extract each specification to check included in the .smv file. As $\verb|prop.expr|$ extracts specifications that are not limited for invariant checking, there is a constraint in place in the Python script to skip these specifications. For each specifications, the function $\verb|prop.not_|$ computes the negation. Finally, we can get the sets of states of \texttt{bddfsm} satisfying $\verb|nspec|$ as a BDD by calling the function \texttt{SPEC\_TO\_BDD}, as shown in the snippet below.

\medskip

\begin{algorithmic}[1]
\State $nspec \leftarrow$ prop.not\_$(spec)$
\State $bddspec \leftarrow$ SPEC\_TO\_BDD$(bddfsm, nspec)$
\end{algorithmic}

\medskip

By computing \texttt{bddspec}, we found the BDD of states that satisfy \texttt{nspec}. However, if \texttt{spec} is an invariant, then no state contained in \texttt{bddspec} should be reachable from the initial states. That is the reason why we chose to compute the reachability BDD (a.k.a. the BDD of reachable states starting from the initial states) and then intersect them with \texttt{bddspec}: if the result is an empty BDD, then states in which \texttt{nspec} is true are never reached. If that is the case, then \texttt{spec} is an invariant.

\subsection{Invariant Check and Reachable States} \label{subsec:invcheck}

By definition, a specification $\varphi$ over state variables is an invariant of the transition system if every reachable state satisfies $\varphi$. In other words, every reachable state must be true according to the invariant specification.

\medskip

Using the definition of an invariant specification, we can start building our algorithm by finding all reachable states for any given SMV models. Trivially, initial states are reachable states. By using the initial states specified in the SMV models, the $\verb|post|$ function of $\verb|fsm|$ module can be used to discover the states within the Post Image of the existing BDD structure, namely all states which can be reached from the initial states according to the SMV model, any states from this Post Image which are not members of the initial states are added to the set of reachable states. The remaining reachable states can then be found by recursively applying the $\verb|post|$ function to the current set of reachable states within the BDD structure. Simialrly as above, any newly discovered states are then added to this set/image of reachable states until no new states are found. The final image is the full set of reachable states within a SMV model.

\medskip

\begin{algorithmic}[1] \label{algo1}
\Function{REACH}{$bddfsm, init$}
    \State $reach\leftarrow init$
    \State $new\leftarrow$ POST$(bddfsm, reach)$
    \While{$new\not=$ INTERSECTION$(reach, new)$}
        \State $reach\leftarrow$ UNION(DIFF$(new, reach),reach$)
        \State $new\leftarrow$ POST$(bddfsm, reach)$
    \EndWhile
    \State \textbf{return }$reach$
\EndFunction
\item[]
\State \textbf{PRE} := $bddfsm$ is the BDD of the system's FSM, $init$ is the BDD containing the initial states.
\State $reach\leftarrow$ REACH$(bddfsm, init)$
\State \textbf{POST} := $reach$ contains the BDD of all reachable states of the SMV Model.
\end{algorithmic}

\subsection{Invariant Specifications} \label{subsec:inv}

Recall that if a specification is an invariant for a SMV model, then all of its reachable states respect (hold true to) the specification. Therefore, no reachables states should be contained in the BDD returned by $\verb|SPEC_TO_BDD|$. This is because that BDD represents the states which are not true to the specification. In other words, if the intersection between the set of reachable states and the BDD created by $\verb|prop.not_|$ is empty, then the specification can be concluded as an invariant. The function will return a tuple $\verb|(True, None)|$, with $\verb|True|$ meaning that the specification of interest is an invariant.

\medskip

\begin{algorithmic}[1]
\If{INTERSECTION$(bddspec, reach) = \emptyset$}
	\State \textbf{return }$(True, None)$
\EndIf
\item[]
\State \textbf{PRE} := $init$ is the set of initial states of $bddfsm$, $bddspec$ is the BDD of states of $bddfsm$ satisfying $nspec$, $reach$ contains the BDD of all reachable states of $bddfsm$.
\State \textbf{POST} := returns $True, None$ if no reachable state starting from  
$init$ is contained in $bddspec$, i.e. if $spec$ is an invariant.
\end{algorithmic}

\subsection{Non Invariant Specifications} \label{subsec:noninv}

However, if the intersection is not empty, then there is at least one reachable state which does not hold true for the specification. In this case, a counterexample is needed as evidence. We can construct a counterexample as follows.

\begin{enumerate}
	\item Randomly select a reachable state which violates the specification, store this state.
	\item Find the pre-image of this state, that is, the set of states that could lead to the selected reachable state in 1 step, using the built-in \texttt{pre} function.
	\item Store the pre-image found.
	\item Find the pre-image of the current pre-image, that is, the set of states that could lead to the selected imagine stored in 3.
	\item Repeat step 3 and 4, until we find an initial state.
\end{enumerate}

\medskip

\begin{algorithmic}[1]
\State $counter\_examples \leftarrow$ INTERSECTION$(bddspec, reach)$
\Function{BACKWARD\_IMAGE\_COMP}{$bddfsm, init, counter\_examples$}
    \State $images \leftarrow \emptyset$
    \State $pre\_counter\_example \leftarrow$ PICK\_ONE\_STATE\_RANDOM$(counter\_examples)$
    \State APPEND$(images, pre\_counter\_example)$
    \While{INTERSECTION$(init, pre\_counter\_example)\not=\emptyset$}
        \State $counter\_example \leftarrow pre\_counter\_example$
        \State $pre\_counter\_example \leftarrow$ PRE$(bddfsm, counter\_example)$
        \State INSERT$(images, pre\_counter\_example)$
    \EndWhile
    \State \textbf{return }$images, counter\_example\_original$
\EndFunction
\item[]
\State \textbf{PRE} := $bddfsm$ is the BDD of the system's FSM, $init$ is the BDD containing the initial states, $counter\_examples$ contains the set of all states in the SMV Model which do not respect the LTL specification.
\State $images\leftarrow$ BACKWARD\_IMAGE\_COMP$(bddfsm, init, counter\_examples)$
\State \textbf{POST} := $images$ contains a list of images where each image contains a set of states which are the preiamges of the next image. $counter\_example\_original$ is the random counter example selected by this algorithm.
\end{algorithmic}

\medskip

$images$ contains the path of states from the initial state of the SMV model, to the randomly selected reachable state which violates the specification for variant checking. The final step of our algorithm is to find the inputs between each interim state in this path, we can construct this as follows:

\begin{enumerate}
    \item Start from the initial state, we can compute the post image of the state by using the built-in $\verb|post|$ function.
    \item Find the intersection between this post image and the second image of $images$, as this is the "next" state which will lead to our counterexample state. Store this state.
    \item Find an input required to go from the initial state to this intersection by utilising the functions $\verb|GET_INPUTS_BETWEEN_STATES|$ and $\verb|PICK_ONE_INPUTS|$. Store this input set.
    \item Similar to step 2, find intersection between the post image of the current state and the next image of $images$. Store this state.
    \item Similar to step 3, find a possible input required to go from the current state to this intersection. Store this input set.
    \item Repeat step 4 and 5, until we reach to the counterexample.
\end{enumerate}

\newpage

\begin{algorithmic}[1]
\Function{FIND\_TRACE}{$bddfsm, init, images, counter\_example\_original$}
    \State $trace \leftarrow \emptyset$
    \State $start \leftarrow init$
    \For{$i\leftarrow1$ $\bf{to}$ $n$} \Comment{LENGTH(images - 1)}
        \State $start \leftarrow$ INTERSECTION$(start, images[i])$
        \State $next\_state \leftarrow$ PICK\_ONE\_STATE$(start)$
        \State APPEND$(trace, next\_state)$
        \State $post \leftarrow$ INTERSECTION(POST$(start), images[i+1])$
        \State $inputs \leftarrow$ GET\_INPUTS\_BETWEEN\_STATES$(start, post)$
        \State APPEND$(trace,$ PICK\_ONE\_INPUTS$(inputs))$
        \State $start \leftarrow post$
    \EndFor
    \State APPEND$(trace, counter\_example\_original)$
    \State \textbf{return }$trace$
\EndFunction
\item[]
\State \textbf{PRE} := $bddfsm$ is the BDD of the system's FSM, $init$ is the BDD containing the initial states, $images$ is the output from BACKWARD\_IMAGE\_COMP function, $counter\_example\_original$ is the random counter example selected by the function BACKWARD\_IMAGE\_COMP, which is equivalent to $images[n]$, where $n$ is the index of the last member of $images$.
\State $trace\leftarrow$ FIND\_TRACE$(bddfsm, init, images, counter\_example\_original)$
\State \textbf{POST} := $trace$ contains the states and the transitions (inputs) which constitutes the path to get from an initial state to the randomly chosen counterexample.
\end{algorithmic}

\medskip
Finally, the set of states and inputs can be returned by the function $\verb|check_|$\\$\verb|explain_inv_spec|$ which shows a counterexample of how a reachable state of the model invalidates the specification. Starting from the initial state, then the first set of inputs, then to the next state, second set of inputs, and repeat until the counter example of reachable state is listed.

\medskip

\begin{algorithmic}[1]
\If{INTERSECTION$(bddspec, reach) \not= \emptyset$}
	\State $counter\_examples \leftarrow$ INTERSECTION$(bddspec, reach)$
    \State $images,counter\_example\_original \leftarrow$ \newline
            \hspace*{5em}BACKWARD\_IMAGE\_COMP$(bddfsm, init, counter\_examples)$
    \State $trace \leftarrow$ \newline
            \hspace*{5em}FIND\_TRACE$(bddfsm, init, images, counter\_example\_original)$
    \State \textbf{return }$(False, trace)$
\EndIf
\item[] 
\State \textbf{PRE} := $bddfsm$ is the FSM of the system represented as a BDD, $init$ is the set of initial states of $bddfsm$, $bddspec$ is the BDD of states of $bddfsm$ satisfying $nspec$, $reach$ contains the BDD of all reachable states of $bddfsm$
\State \textbf{POST} := returns the couple $False, trace$ where $trace$ contains the states and the transitions (inputs) which constitutes the path to get from an initial state to the randomly chosen counterexample.
\end{algorithmic}